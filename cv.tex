% arara: clean: { files: [cv.aux , cv.log]}
% arara: xelatex
% arara: xelatex
% arara: clean: {extensions: [aux,toc,vrb,snm,blg,bcf,bbl,out,aux,lof,lot,xml,log,run.xml,el,nav]}

%%%%%% run compilation as follows:
%%% while true ; do inotifywait -rq --event modify . ; arara cv.tex ; done
%%% and check the log file for errors
%%% watch -n 3 tail -n 50 cv.log

\documentclass{muratcan_cv}


\setname{Dylan}{Festa}
\setmail{dylan.festa@~$\large[$tum.de \emph{or} gmail.com $\large]$}
\setmobile{+49~(0)~176~85453368}
\setgithubaccountlink{https://github.com/dylanfesta}
\setgithubaccountname{github.com/dylanfesta}
\settwitterlink{https://x.com/dylanfesta}
\settwittername{dylanfesta}

\setthemecolor{ForestGreen} %you can play with color of the template (red is also nice..)

\begin{document}
%Set variables
%You can add sections, texts, explanations just by copying the style below. Replace the dummy texts "\lipsum[1][x-x]\par" with actual texts.
%Create header
\headerview

\vspace{1ex}
%Sections
%
% Summary
% \addblocktext{Current Position}{%
% \datedexperience{Postdoctoral Researcher}{2021-present}
% \explanation{Gjorgjieva group, Max Plank Institute for Brain Research, Frankfurt am Main, Germany.}
% \explanationdetail{I do stuff}
% }
% %
\section{Current Position}{%
 \datedexperience{Postdoctoral Researcher}{2021-present}
 \explanation{Lab of Prof.~Julijana Gjorgjieva, School of Life Science, Technical University of Munich $\large($TUM$\large)$, Germany.}
\explanationdetail{
 \coloredbullet\ %
   Using both numerical and analytic approaches, I study how inhibitory spike-timing dependent plasticity rules can ensure excitatory/inhibitory balance, while also generating specific connectivity structures in recurrent circuits. I also mentor students taking part in internships, with projects related to plasticity and neural dynamics, and help as  teaching assistant (Computational neuroscience course, Master of Science in Neuroengineering, TUM).}
 }
%
\section{Research Experience}{%
 \datedexperience{Postdoctoral Researcher}{2016-2021}
 \explanation{Lab of Dr.~Ruben Coen-Cagli, Albert Einstein College of Medicine, Bronx, NY, USA}
 \explanationdetail{
 \coloredbullet\ %
  I  developed a probabilistic inference model tuned to natural image statistics, to make predictions on both mean response and trial-to-trial response variability of neurons in primary visual cortex (V1). Computationally, I reimplemented and extended the model of interest, finding new results that I used for both analytical studies and numerical simulations. In parallel to this, I worked in close collaboration with experimentalists in the laboratory of Prof.~Adam Kohn group, to test the theory using both existing data and new experiments. I contributed to both stimulus design and data analysis (including spike-sorting).
 }
%
  \datedexperience{Ph.D.~Studies}{2012-2016}
 \explanation{Group of Prof. M\'at\'e Lengyel, Computational and Biological Learning Lab (CBL), Cambridge, UK} 
 \explanationdetail{
  \coloredbullet\ %
 I built from scratch a novel optimisation procedure that operates on stabilized supralinear networks (SSNs) with multiple fixed-point attractors. The optimized networks display several relevant biological features, and can robustly store and recall memory state that correspond to population firing patterns. In my work, I modeled and tested these systems extensively, and compared them with state-of-the-art approaches used to construct attractor networks.\par
 }
%
 \datedexperience{Research Intern}{2011-2012}
 \explanation{lab of Prof.~Fred Wolf, MPI for Complex Dynamics and Self Organization G\"ottingen, Germany}
 \explanationdetail{
  \coloredbullet\ %
 Recurrent neural networks (RNNs) can be in a chaotic dynamical regime. In this project,
starting from a RNN of quadratic integrate and fire neurons, I implemented the computation of the so-called “Lyapunov vectors”, that measure not only the presence of chaos, but also the direction and dimensionality of the most dynamically stable or unstable dimensions. This allows for a more in-depth characterization of the ergodic properties of RNNs.\par
 }
 }

\section{Publications}
    \datedexperience{Formation and computational implications of assemblies in neural circuits}{2022}
    \explanation{%
     Christoph Miehl, Sebastian Onasch, \textbf{Dylan Festa}, Julijana Gjorgjieva}
    \explanationdetail{
    The Journal of Physiology,%
    \href{https://doi.org/10.1113/JP282750}{DOI: 10.1113/JP282750}}
    %
    \datedexperience{Adaptation of Drosophila larva foraging in response to changes in food resources}{2022}
    \explanation{%
     Marina Wosniack, \textbf{Dylan Festa}, Nan Hu, Julijana Gjorgjieva, Jimena Berni}
    \explanationdetail{
    eLife,%
    \href{https://doi.org/10.7554/eLife.75826}{DOI: 10.7554/eLife.75826}}
    %
    \datedexperience{Neuronal variability reflects probabilistic inference tuned to natural image statistics}{2021}
    \explanation{\textbf{Dylan  Festa}, Amir Aschner, Aida Davila, Adam Kohn, Ruben Coen-Cagli}
    \explanationdetail{
    Nature Communications,%
    \href{https://doi.org/10.1038/s41467-021-23838-x}{DOI: 10.1038/s41467-021-23838-x}}
    %
    \datedexperience{Analog Memories in a Balanced Rate-Based Network of E-I Neurons}{2014}
    \explanation{\textbf{Dylan Festa}, Guillaume Hennequin, Máté Lengyel} 
    \explanationdetail{
Advances in Neural Information Processing Systems 
\href{https://proceedings.neurips.cc/paper/2014/file/0a0a0c8aaa00ade50f74a3f0ca981ed7-Paper.pdf}{(NeurIPS)}}
%
\section{Teaching and Supervision Experience} 

    \datedexperience{Computational neuroscience course}{2023}
    \explanation{Teaching assistant $\large($TA$\large)$ for a university course on computational neuroscience, Technical University of Munich.}
    \explanationdetail{I led the exercise sessions of the course, and helped ideating and grading the final written exams.}
    \datedexperience{Cajal summer school}{2022}
    \explanation{Teaching assistant $\large($TA$\large)$ for the Cajal summer school, Champalimaud center of the Unknown, Lisbon, Portugal} 
    \explanationdetail{I assisted students during tutorials (first week), and then mentored two smaller groups for project work (following two weeks).}
    \datedexperience{NeuroMatch Academy}{2021} 
    \explanation{TA for the first Neuromatch virtual summer school in Computational Neuroscience} 
    \explanationdetail{
    Quality-assessment of the course material before the beginning of the course, and TA during the course, assisting 10 to 15 students as they worked on their tutorial notebooks.
    }
    %
    \datedexperience{Master Thesis Supervision}{2021}
    \explanation{Supervision of Master student}
     \explanationdetail{I supervised the Master student Vanshika Kapoor, for her thesis with title: ``Inferring activity statistics in Stabilized Supralinear Networks''} 
    %
    \datedexperience{Student supervisions}{2014}
    \explanation{University of Cambridge, UK} 
    \explanationdetail{
    During the full extent of my Ph.D.~in Cambridge I tutored small groups of students in a module of the 3rd year course “Introduction to Neuroscience”, held at the Engineering Department}
%
%Education
\section{Education} 
    \datedexperience{Ph.D.~in Computational Neuroscience}{2012-2016} 
    \explanation{Group of Prof.~M\'at\'e Lengyel, Computational and Biological Learning Lab (CBL), Cambridge, UK} 
    %
    \datedexperience{Master degree in Physics}{2008-2011} 
    \explanation{Università degli Studi di Pisa, Pisa, Italy} 
    \explanationdetail{\coloredbullet\ %
    \textbf{Final grade:} 110/110 \emph{cum laude}. \ \  \textbf{Specialization:} Physics of Condensed Matter \par 
     }
    \datedexperience{Bachelor degree in Physics}{2005-2008} 
    \explanation{Università degli Studi di Pisa, Pisa, Italy} 
    \explanationdetail{\coloredbullet\ %
    \textbf{Final grade:} 110/110 \emph{cum laude}. \par 
     }
     \datedexperience{Diploma}{2000-2005} 
    \explanation{Liceo Scientifico D.~Alighieri, Matera, Italy} 
    \explanationdetail{\coloredbullet\ %
    \textbf{Final grade:} 100/100. \par 
     }
%     
% Skills
\section{Skills}
    %
    \newcommand{\skillone}{\createskill{Programming Languages}{\textbf{\emph{Experienced:}} \ \  Julia \cpshalf Python \cpshalf Matlab \ \ \textbf{\emph{Familiar:}} \ \  \LaTeX \cpshalf OCaML  \cpshalf git \cpshalf Linux systems }}
    %
    %
    \newcommand{\skilltwo}{\createskill{Software}{ Adobe Illustrator \cpshalf Microsoft Office \cpshalf Plexton Offline Spike Sorter }}
    %
    \newcommand{\skillthree}{\createskill{Spoken Languages}{\textbf{\emph{Native:}} \ \  Italian \ \ \textbf{\emph{Fluent:}} \ \ English \ \ \textbf{\emph{Beginner:}} \ \  German }}
    %
    \createskills{\skillone, \skilltwo,\skillthree} %, \skillfour, \skillfive}
%
% Experience
\section{Extra}
    \newcommand{\extraone}{%
    \textbf{2018-2020}  \  \ Co-chair of the Einstein Postdoctoral Association (EPA), Einstein College, Bronx, NY. \par
    }
    %
    \newcommand{\extratwo}{%
    \textbf{2018}   \  \  NET Impact NYC Service Corps   \ -- \ 
I volunteered as a pro-bono consultant for a few months, helping a Bronx association to market a K12 school program. The program was based on indoor school farming using hydroponics \par }
    \newcommand{\listofextras}{\extraone, \extratwo}
    %
    \createbullets{\listofextras}
%

% 2018-2020 
% Leadership role in the local postdoctoral organization: I helped organizing and planning social events,
% and I sat in different committees and in the Senate of the college.
% 2018 NET Impact NYC Service Corps.
% I volunteered as a pro-bono consultant for a few months, helping a Bronx association marketing a K12
% school program. The program was centered on indoor school farming, based on hydroponics

%Footnote
\createfootnote
\end{document}
